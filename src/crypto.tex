\documentclass{beamer}

\usepackage{natbib}

\usepackage[frenchb]{babel}

\usepackage[T1]{fontenc}

\usepackage[utf8]{inputenc}

\usepackage{amsmath}

\usepackage{tcolorbox}

\usepackage{lipsum}

\usepackage[labelformat=empty]{caption}

\usepackage{cclicenses}

\usetheme{Darmstadt}

\title{Hygiène cryptographique}

\author{\cc Ilan 'trog' Dubois}

\AtBeginSection[]
{
    \begin{frame}
        \frametitle{Sommaire}
            \tableofcontents[currentsection]
    \end{frame}
}

\begin{document}
    \begin{frame}
        \titlepage
    \end{frame}
    \section{Mb dszquphsbqijf}
    \subsection{Définitions}
        \begin{frame}{label=vocabulaire}
            \frametitle{Vocabulaire}
            \begin{center}
                \begin{itemize}
                    \item \textbf{Chiffrer}: Rendre incompréhensible un messgae pour qui n'aurait pas la clé.
                    \item \textbf{Déchiffrer}: Rendre à un message sa forme originale à l'aide de la clé.
                    \item \textbf{Décrypter}: Rendre à un message sa forme originale sans utiliser de clé (casser le code).
                    \begin{tcolorbox}[colback=green!5,colframe=green!40!black,title=Mais crypter alors ?]
                      En anglais chiffrer se dit \textit{encrypt}, d'où la confusion courante avec le terme \textit{crypter} qui en français signifie mettre dans une crypte.
                    \end{tcolorbox}
                \end{itemize}
            \end{center}
        \end{frame}
        \begin{frame}{label=utilisation}
            \frametitle{Les cas d'utilisation}
            \begin{center}
                \begin{itemize}
                    \item Rendre incompréhensible un document quelconque pour toute personne n'ayant pas la clé.
                    \item Assurer l'intégrité d'un document.
                    \item Assurer l'authenticité d'un document.
                \end{itemize}
            \end{center}
        \end{frame}
    \subsection{Deux principaux types de cryptographie}
        \begin{frame}{label=symmetric}
            \frametitle{La cryptographie symmétrique}
            \begin{center}
                \begin{itemize}
                    \item Chiffrement et déchiffrement se font avec une même clé.
                    \item César, Vigenaire, AES...
                    \item Est très rapide, utilisé pour chiffrer son disque, sa connexion...
                \end{itemize}
            \end{center}
        \end{frame}
        \begin{frame}{label=asymmetric}
            \frametitle{La cryptographie asymmétrique}
            \begin{center}
                \begin{itemize}
                    \item Utilise une paire de clé \textit{public}/\textit{privé}.
                    \item Le chiffrement se fait avec une clé publique. Le déchiffrement nécessite la clé privée.
                    \item DSA, RSA, Ed25519...
                    \item Plus lent mais aussi très pratique dans le cas où les parties ne partagent pas encore de \textit{secret}. Sert ainsi souvent à initier une connexion avec un chiffrement symmétrique.
                \end{itemize}
            \end{center}
        \end{frame}
        \appendix
        \begin{frame}
            \bibliography{src/tracking}{}
            \bibliographystyle{plain}
        \end{frame}
\end{document}
